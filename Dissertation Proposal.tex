\documentclass[english,nohyper,nofonts,nobib,nols,twoside]{tufte-handout}
\usepackage{fontspec}
\usepackage{titlesec}
\usepackage{ctable}
\usepackage[unicode=true]{hyperref}
\usepackage{amsmath}
\usepackage{etoolbox}% provides some support for comma-separated lists
\usepackage{xunicode}
\usepackage{polyglossia}
\setdefaultlanguage{english}
\usepackage{graphicx}
\usepackage{color}
\usepackage{listings}
%\usepackage{doi}
%\usepackage{float}

%%% General formatting options

\newfontfamily\headingfont[Contextuals=NoLineFinal, ItalicFeatures={Contextuals=NoLineFinal}]{Hoefler Text}


\setmainfont[Mapping=tex-text]{Georgia}
\setsansfont[Mapping=tex-text]{Helvetica Neue Light}
\setmonofont{Monaco}

\setlength{\parskip}{\smallskipamount}
\setlength{\parindent}{0bp}

\titleformat{\section}%
  [hang]% shape
  {\headingfont\LARGE}% format applied to label+text
  {\thesection}% label
  {1em}% horizontal separation between label and title body
  {}% before the title body
  []% after the title body

%\titlespacing*{\section}{0pt}{0pt}{0pt}[]

\titleformat{\subsection}%
  [hang]% shape
  {\headingfont\Large\itshape}% format applied to label+text
  {\thesubsection}% label
  {1em}% horizontal separation between label and title body
  {}% before the title body
  []% after the title body

\bibliographystyle{/Users/noamross/Dropbox/Public/pd/ecology}

%%% Code Highlighting



%%% Macro re-writes

\makeatletter
% We'll keep track of the old/seen bibkeys here.
\def\@tufte@old@bibkeys{}

% This macro prints the full citation if it's the first time it's been used
% and a shorter citation if it's been used before.
%\newcommand{\@tufte@print@margin@citation}[1]{%
%  % print full citation if bibkey is not in the old bibkeys list
%  \ifinlist{#1}{\@tufte@old@bibkeys}{%
%    \citealp{#1}% print short entry
%  }{%
%    \bibentry{#1}% print full entry
%  }%
%  % add bibkey to the old bibkeys list
%  \listgadd{\@tufte@old@bibkeys}{#1}%
%}

% We've modified this Tufte-LaTeX macro to call \@tufte@print@margin@citation
% instead of \bibentry.
%\renewcommand{\@tufte@normal@cite}[2][0pt]{%
%  % Snag the last bibentry in the list for later comparison
%  \let\@temp@last@bibkey\@empty%
%  \@for\@temp@bibkey:=#2\do{\let\@temp@last@bibkey\@temp@bibkey}%
%  \sidenote[][#1]{%
%    % Loop through all the bibentries, separating them with semicolons and spaces
%    \normalsize\normalfont\@tufte@citation@font%
%    \setcounter{@tufte@num@bibkeys}{0}%
%    \@for\@temp@bibkeyx:=#2\do{%
%      \ifthenelse{\equal{\@temp@last@bibkey}{\@temp@bibkeyx}}{%
%        \ifthenelse{\equal{\value{@tufte@num@bibkeys}}{0}}{}{and\ }%
%        \@tufte@trim@spaces\@temp@bibkeyx% trim spaces around bibkey
%        \@tufte@print@margin@citation{\@temp@bibkeyx}%
%      }{%
%        \@tufte@trim@spaces\@temp@bibkeyx% trim spaces around bibkey
%        \@tufte@print@margin@citation{\@temp@bibkeyx};\space
%      }%
%      \stepcounter{@tufte@num@bibkeys}%
%    }%
%  }%
%}


% Calling this macro will reset the list of remembered citations. This is
% useful if you want to revert to full citations at the beginning of each
% chapter.
\newcommand{\resetcitations}{%
  \gdef\@tufte@old@bibkeys{}%
}

\renewcommand{\maketitle}{%
  \newpage
  \global\@topnum\z@% prevent floats from being placed at the top of the page
  \begingroup
  \setlength{\parindent}{0pt}
  \setlength{\parskip}{8pt}
  \begin{fullwidth}
  \begin{center}
   \begingroup
      \par{\headingfont\huge{\@title}}
      \ifthenelse{\equal{\@author}{}\OR\equal{\@date}{}}{
      \ifthenelse{\equal{\@author}{}}{}{\par{\headingfont\Large\textit{\@author}}}
      \ifthenelse{\equal{\@date}{}}{}{\par{\headingfont\Large\textit{\@date}}}}{\par{\headingfont\Large\textit{\@author, \@date}}}
      \par{\headingfont\Large\vspace{3em}}
  \endgroup
  \end{center}
  \end{fullwidth}
  \endgroup
  \thispagestyle{plain}% suppress the running head
}

% The 'fancy' page style is the default style for all pages.
\fancyhf{} % clear header and footer fields
\ifthenelse{\boolean{@tufte@twoside}}
  {\fancyhead[LE]{\thepage\quad\headingfont\scshape{\newlinetospace{\plainauthor}}}%
    \fancyhead[RO]{\headingfont\scshape{\newlinetospace{\plaintitle}}\quad\thepage}}
  {\fancyhead[RE,RO]{\headingfont\scshape{\newlinetospace{\plaintitle}}\quad\thepage}}



%%Re-size images to fit text width
\def\maxwidth{\ifdim\Gin@nat@width>\linewidth\linewidth
\else\Gin@nat@width\fi}

\newlength{\fullwidthlength}
\AtBeginDocument{\setlength{\fullwidthlength}{\@tufte@fullwidth}}

\makeatother

\let\Oldincludegraphics\includegraphics
\renewcommand{\includegraphics}[1]{\Oldincludegraphics[width=\maxwidth]{#1}}




%\renewenvironment{quote}{\begin{tabular}{|p{0.9\textwidth}}{\\%
%\end{tabular}}
%\let\oldcitep\citep
%\renewcommand{\citep}[1]{\oldcitep{#1}\cite{#1}}
\renewcommand\citep\cite
\let\oldcitet\citet
\renewcommand{\citet}[1]{\oldcitet{#1}\cite{#1}}

\renewcommand\subsubsection\paragraph

%%% Set metadata

\title{Forest Disease Dynamics and Management}
\author{Noam Ross}
\date{13 January 2022}

%%%% Document body
\definecolor{lightlightgray}{gray}{0.95}
\begin{document}
\lstset{language=R,%
basicstyle=\ttfamily\small,%
linewidth=\fullwidthlength,%
xleftmargin=8pt,%
xrightmargin=8pt,%
backgroundcolor=\color{lightlightgray},%
breaklines=true,%
showspaces=false,%
breakindent=18pt,%
}

\maketitle

\section{Abstract}

Forest disease spreads through plant communities structured by species
composition, age distribution, and spatial arrangement. I propose to
examine the consequences of the interaction of these components of
population structure on a model systems of \emph{Phytophthora ramorum}
invasion in California redwood forests. First, I will compare the
dynamic behavior of a series of epidemiological models that include
different configurations of population structures. Then I will fit these
models to time-series data from a network of disease monitoring plots to
determine what components of forest population structure are most
important for prediction of disease spread. Using the most parsimoniuous
models, I will determine optimal schedules of treatment to minimize the
probability of disease outbreak.

\section{Introduction: Modeling Disease in Complex Populations for
Management}

Forest diseases can radically transform ecosystems. In North America,
chestnut blight, Dutch elm disease and beech bark disease have caused
precipitous declines in their hosts, leading to changes in the structure
and function of forests. Changes from such diseases may be the dominant
force changing the face of some forests in coming decades, overwhelming
other forms of rapid environmental change such as climate change
\citep{Lovett2006}.

Even simple host-disease systems exhibit complex dynamics
\citep{Kermack1927}. Forest disease dynamics include additional
complexity driven by variation in pathogen and host populations, spatial
strucure, pathogen life cycles, and demographic and environmental
stochasticity \citep{Hansen2000, Holdenrieder2004}. These factors yield
greater complexity in disease dynamics. This complexity, combined with
limited monitoring data and the uncertainty of the parameters of
emergent diseases, makes prediction of disease dynamics difficult. Yet
such prediction is needed decision-making in allocating limited
resources for treatment and prevention.

One area of complexity in forest systems is structured variation in host
populations. Host tree populations may consist of multiple species or
genotypes, and consist of multiple sizes, each of which may interact
with disease or each other differently \citep{Gilbert1996}. All forest
tree populations are spatially structured; non-motile individuals can
not be ``well-mixed'' \citep{Filipe2003}. Each of these forms of
population structure have consequences for population dynamics that have
been examined in theoretical models
\citep{Dobson2004, Klepac2010, Park2001, Park2002}. However, dynamics
resulting from these components of population structure interacting are
not well-characterized.

Understanding modeled dynamics of disease in structured populations may
provide insights useful for management. However, the ability to use such
models for management planning depends requires confidence in their
predictive power, particularly under novel management treatments. Recent
developments in \emph{particle filter} techniques allow estimation of
the likelihoods of complex dynamic models
\citep{Arulampalam2002, Ionides2006, Knape2012}, and subsequent
comparison of models in out-of-sample predictive performance
\citep{Vehtari2012}. Fitting the dynamic models in full provides more
confidence in the dynamic results than deriving madel parameters
individually from data, because it tests whether models' emergent
dynamics are supported by the data.

Modeling multidimensional population structure is problematic for
several reasons. First, as models get increasingly complex, analytic
solutions are less likely to be tractable, and complete characterization
of model behavior is unlikely. Thus model (and) system behavior is more
difficult to explore. Secondly, high-dimensional models are difficult to
estimate; each additional form of population structure multiplies the
parameters involved, and requires additional data.

Sudden oak death (SOD) is an emerging forest disease in California and
Oregon that threatens populations of tanoak (\emph{Notholithocarpus
densiflorus}) \citep{Rizzo2003}, and has the potential to modify
community structure \citep{Metz2012}, and cause significant economic
damage \citep{Kovacs2011}. Silvicultural chemical, and other control
techniques can modify the progression of disease \citep{Swiecki2013},
but eradication of SOD is unlikely {[}@Cobb2013{]} even at local scales.
Long and short- and long-term solutions to limit the damage of disease
are needed will require continuous treatment and monitoring regimes.
Under budget constraints, planning for such treatment can be informed by
dynamics models via optimal control techniques.

I aim to answer the following overall questions:

\begin{enumerate}
\def\labelenumi{\arabic{enumi}.}
\itemsep1pt\parskip0pt\parsep0pt
\item
  How do interactions between community, size, and spatial structure
  affect disease dynamics in forests?
\item
  What aspects of population structure are most important in determining
  the probability and size of forest disease outbreaks in theoretical
  models?
\item
  What aspects of population structure are most important in
  \emph{predicting} the probability and size of SOD outbreaks?
\item
  What population structure minimizes the risk of SOD outbreak while
  maintaining desirable populations of at-risk species?
\item
  What is the most cost-effective strategy for maintaining populations
  of tree species in the presence of SOD?
\end{enumerate}

\section{Background}

\subsection{Sudden Oak Death}

Sudden Oak Death (SOD) is an emerging forest disease in California and
Oregon that poses risks to forest across North America. First observed
in California in the mid-1990s, SOD is caused by the water mold
\emph{Phytophthora ramorum} \citep{Rizzo2002}. \emph{P. ramorum} often
kills tanoak (\emph{Notholithocarpus densiflorus}), which provide
habitat and food to many vertebrate species, and are the primary host of
ectomycorrhizal fungi in redwood forests \citep{Rizzo2003}. Loss of this
species may have cascading effects on other species. The disease has
also caused significant economic damage through removal costs and
property value reduction \citep{Kovacs2011}. It has the potential to
spread to species in other regions, such as northern red oak
(\emph{Quercus rubra}), one of the most important eastern timber species
\citep{Rizzo2002}.

\subsection{Population Structure: Multiple Host Species}

\emph{Phytophthora ramorum} has over 100 host species in 40 genera,
which fall into several functional types \citep{Swiecki2013}. In canker
hosts, which are all members of \emph{Fagaceae} in California, pathogen
produces cankers in the trunk. Few if any spores are produced from the
cankers, and these hosts are generally dead ends. SOD is fatal to many
canker costs. In foliar hosts, \emph{P. ramorum} resides and reproduces
in leaves and twigs. Foliar hosts vary greatly in suscepibility to
disease and spore production from diseased individuals. Some hosts are
dead ends; others are major drivers of disease spread and survival.

In Redwood forests in Northern California, the hosts of importance are
tanoak and bay laurel (\emph{Umbellularia californica}). Tanoak has the
dubious distinction of being the only species that is both a canker and
foliar host, and the disease is generally lethal in these trees. Bay
laurel is a foliar host in which \emph{P. ramorum} produces
prolifically, and it suffers no harm from the disease \citep{DiLeo2009}.
Redwood and other species are unsusceptible or dead-end hosts where
\emph{P. ramorum} has little detrimental effect. In forests of the
redwood-tanoak-bay complex, bay Laurel acts as the primary disease
reservoir {[}@Davidso2008{]}. Tanoak infection and mortality is
dependent on bay laurel density \citep{Cobb2012}, and removal of laurel
is a commonly suggested treatment to reduce risk of SOD
\citep[@Filipe2013;][]{Swiecki2013}.

\citet{Dobson2004} examined a model of disease in multiple host species
based on an the SIR \citep{Kermack1927} framework. In it, species had
density dependence but interacted only though disease transmission.
Importantly, transmission between species was always equal or less than
transmission between species. Dobson found that in when transmission was
density-dependent, species had a complementary effect on the disease's
rate of reproduction $(R_0)$; that is, a mix of both species led to
faster outbreaks. However, in frequency-dependent case, $R_0$ was
greatest when one species dominated. Transmission between species
dampens and synchronizes disease oscillations in each species, and at
very high contact rates between species the most susceptible species are
more like to go extinct. \citet{Craft2008} implemented a stochastic
version of this model in a system of mamallian carnivores, finding that
species with low intra-species contact rates acted as sinks for disease,
and their contact with more social species increased spread disease
within these low-contact species.

SOD dynamics differ from these systems in a few important ways. First,
the contact structure among species may not be symmetrical, and
between-species contacts may be as high as within. Contact rates in this
case are driven by differences in species transmissivity (amount of
spore produced) and susceptibility (probability of infection per spore)
rather than social grouping. Where one species with high transmissivity
coexists with a species of high susceptibility, the probability of
transmission may be higher between species than within. Also, the
spatial structure of species, and the dispersal pattern of spores, may
result in a hybrid between density and frequency-dependent disease
transmissions.

\subsection{Population Structure: Size/Stage Classes}

Within the tanoak population, epidemiological characteristics vary with
tree size. Trees of larger size classes are more likely to be infected
and die more quickly than smaller tree \citep{Cobb2012}, Possibly due to
the vulnerability of cracked bark and the amount of bark tissue
available for invasion \citep{Swiecki2005}. There is no evidence
currently for differences in spore production or other physiological
effecst across tree sizes {[}@Davidson2008{]}.

When disease dynamics occur much faster than demographic processes, we
can treat size classes similarly to different species, and examine the
progression of disease in a constant population structure. However, when
disease progression and growth occur at similar rates, these processes
can interact to produce complex dynamics. This is the case with SOD;
infectious periods can last many years, during which trees may continue
to grow.

The effects of age-based population structure and contact rates have
been studied extensively in the context of human disease and vaccination
programs \citep[\citet{Metcalf2011}]{Anderson1985}. Reducing in-class
transmission rates at young ages increases the average age of infection,
independent of the exact contact structure. When only the
susceptibility, and not transmissivity, of each age class varies,

\citet{Klepac2010} created a framwork for examining disease dynamics in
stage-structured populations. In this, disease and epidemiological
processes are alternate in time (TODO: \emph{periodic something}). This
approximation holds with SOD in California, where most sporulation
occurs during the winter rainy months, while tree growth occurs in the
spring and fall {[}TODO:ref{]}.

\citet{Klepac2010} found that increases in within-class transmission
increases both the infected and recovered population, relative increases
in that class

Protecting a class drives average age at infection to other classes. If
infection is most common in the young, then vaccination increases age to
infection. Vaccination in the old will, too, but much less. Accentuates
oscillation. ˜

In SOD, contact rates appear to depend less on mixing rates. But
transmissivity varies across species, and suscepibility varies across
age classes.

\subsection{Population Structure: Space}

\emph{P. Ramorum} dispersal interacts with host population structure at
many scales. The pathogen spreads between trees via wind-blown rain, and
splash, limiting most spores to spread within 15m of host plants
\citep{Davidson2005}. However, occaisional weather events, such as fog,
can transport spores up to 3 km \citep{Rizzo2005}. Over long distances,
\emph{P. ramorum} can be transported in streams or spread via
human-mediated vectors such as nursery plant trade
{[}@Osterbauer2004{]}. \citet{Meentemeyer2011} found that the best fit
kernel for \emph{P. ramorum spread} was the sum of two Cauchy kernels,
one on the scale of tens of meters and another on the scale of tens of
kilometers.

Despite the ability to spread long distance, strong meter-scale
gradients mean that individual trees in stands can not be considered
``well mixed'' in terms of contact rates between trees. Rather, contact
patterns across space arise from the interaction of spatial clustering
of trees and the dispersal kernel of the disease. Neither frequency- nor
density-dependent transmission characterizes this arrangement.

The effect on such spatial structure on development of epidemics has
been studied with continuous populations in space {[}@Bolker1999{]},
metapopulation models (\citet{Park2001}; \citet{Park2002}), models of
individuals on a lattice \citep{Filipe2003, Filipe2004} and discrete
individuals in continuous space (\citet{Brown2004a}). These approaches
have reached similar conclusions of the effect of spatial structure in
non-mobile populations. In all cases, the threshold of disease growth
for a global outbreak $(R_0)$ is greater than the threshold for a local
epidemic $(R_L)$. For instance, in metapopulations on a lattice,
$R_0 = R_L (1 + z\varepsilon)$, where $z$ is the average number of
connections between patches and $\varepsilon$ is the inter-patch contact
rates \citep{Park2001}. In models of discrete individuals in space,
$R_0 = \lambda(1 + \bar{\mathcal{C}} SI)$, where $\lambda$ would be the
pathogen growth rate in a well-mixed population, and $\bar{\mathcal{C}}$
is the dispersal kernel-weighted spatial correlation between susceptible
and infected individuals. $\bar{\mathcal{C}}$ evolves over time but
reaches a minimum that represents the threshold required for a global
epidemic \citep{Brown2004a}. Also, both clustering and anti-clustering
(oversdispersal) reduce the rate of increase of disease, except when
clustering occurs at the scale of dispersal, in which case it can
accellerate spread. Fat-tailed dispersal kernels accellerate spread, as
well.

\section{Study Approach}

I will examine the comparative importance of these components of
population structure by (1) characterizing the dynamic behavior of
models that include each type of structure and their combinations and
(2) identifying the model structure that best predicts data of disease
spread over time in redwood-tanoak-bay forests. Using the best-fit
model, I determine optimal paths for silvicultural and stand protection
treatments to minimize the risk of outbreak over time

\subsection{Comparative Dynamics}

To characterize the effects of the population structure on disease
dynamics, I will compare four models. All are extensions of the
epidemiological model of \citet{Cobb2012}.

\textbf{Model A} is the simplest of the four models, only representing
structure in the form of differences between species of trees. It is
described by this system of stochastic difference equations:

\[\begin{aligned}
\boldsymbol{S}_{t+1} &\sim \boldsymbol{S_t} + \overbrace{  \text{Pois}\left[\boldsymbol{b(S_t+I_t)}\left(1-\sum_{i=1}^n w_i (S_{it} + I_{it}) \right) + \boldsymbol{rm_I I_t}\right]}^{\text{new recruits}} - \overbrace{\text{Binom}_1\left((\underbrace{1 - e^{-\boldsymbol{\beta I_t} - \boldsymbol{\lambda_{ex}}}}_{\text{force of infection}})\boldsymbol{S_t} \right)}^{\text{infections}} -\overbrace{\text{Binom}_2(\boldsymbol{m_S S_t})}^{\text{mortality}} \\
\boldsymbol{I}_{t+1} &\sim \boldsymbol{I_t} + \text{Binom}_1\left((1 - e^{-\boldsymbol{\beta I_t} - \boldsymbol{\lambda_{ex}}})\boldsymbol{S_t} \right) - \text{Binom}_3(\boldsymbol{m_I} \boldsymbol{S_t}) 
\end{aligned}\]

Here $\boldsymbol{S}_t$ and $\boldsymbol{I}_t$ are vectors of the
populations of susceptible and infected individuals of each species, and
$\boldsymbol{\beta}$ is the matrix of contact rates. Other parameters
are listed below in Table 1.

\begin{longtable}[c]{cl}
\hline\noalign{\medskip}
\begin{minipage}[b]{0.43\columnwidth}\centering
Parameter Vector Symbol
\end{minipage} & \begin{minipage}[b]{0.53\columnwidth}\raggedright
Description
\end{minipage}
\\\noalign{\medskip}
\hline\noalign{\medskip}
\begin{minipage}[t]{0.43\columnwidth}\centering
$\boldsymbol m_{S,I}$
\end{minipage} & \begin{minipage}[t]{0.53\columnwidth}\raggedright
Probability of death per year, for both susceptible and infectious trees
\end{minipage}
\\\noalign{\medskip}
\begin{minipage}[t]{0.43\columnwidth}\centering
$\boldsymbol b$
\end{minipage} & \begin{minipage}[t]{0.53\columnwidth}\raggedright
Fecundity per individual per year.
\end{minipage}
\\\noalign{\medskip}
\begin{minipage}[t]{0.43\columnwidth}\centering
$\boldsymbol r$
\end{minipage} & \begin{minipage}[t]{0.53\columnwidth}\raggedright
Probability of resprouting after death by disease.
\end{minipage}
\\\noalign{\medskip}
\begin{minipage}[t]{0.43\columnwidth}\centering
$\boldsymbol w$
\end{minipage} & \begin{minipage}[t]{0.53\columnwidth}\raggedright
Competitive coefficient (relative contribution to density-dependent
recruitment)
\end{minipage}
\\\noalign{\medskip}
\begin{minipage}[t]{0.43\columnwidth}\centering
$\boldsymbol \lambda_{ex}$
\end{minipage} & \begin{minipage}[t]{0.53\columnwidth}\raggedright
Rate of contact of each species with pathogen spores from outside the
site
\end{minipage}
\\\noalign{\medskip}
\hline
\noalign{\medskip}
\caption{Model A Parameters}
\end{longtable}

New susceptible trees enter the system via density-dependent seedling
recruitment and resprouting from recently killed trees. Susceptible
trees become infected at rates proportional contact with infected trees
(density-dependent) and pathogen migrating into the system. Both
susceptible and infected trees die at constant rates.

The model is modified from \citet{Cobb2012} in several ways: (1)
inclusion of demographic stochasticity, (2) conversion from continuous
discrete time, (3) the inclusion of $\lambda_ex$, force of infection
from areas outside the system, (4) the exclusion of recovery of infected
trees, which has not been observed in the field. Finally, the only
population structure in Model A is the difference in species parameters.
It excludes age structure, and spatial structure, assuming a well-mixed
population and frequency-dependent transmission (Following
\citet{Diekmann1990})

The model will be parameterized with three speces: tanoak, bay, and
redwood, which will represent all non-host species. And run with initial
conditions found in \citet{Cobb2012} and @Filipe2013.

\textbf{Model B} adds stage structure to Model A. In this case, the
vectors $\boldsymbol{S_t}$ and $\boldsymbol{I_t}$ represent the
population divided by species and size class:

\[\begin{aligned}
S'_t &\sim S_t - \text{Binom}_1\left((1 - e^{-\boldsymbol{\beta I_t} - \boldsymbol{\lambda_{ex}}})\boldsymbol{S_t} \right) \\
I'_t &\sim I_t + \text{Binom}_1\left((1 - e^{-\boldsymbol{\beta I_t} - \boldsymbol{\lambda_{ex}}})\boldsymbol{S_t} \right) \\
S'_t &\sim \text{Multinom}(\boldsymbol{A_S(S'_t,I'_t)S'_t}) + \text{Pois}\left[\boldsymbol{b(S'_t+I'_t)}\left(1-\sum_{i=1}^n w_i (S'_{it} + I'_{it}) \right) + \boldsymbol{rm_I I'_t}\right] \\
I'_t &\sim \text{Multinom}(\boldsymbol{A_I(S_t,I_t)I_t})
\end{aligned}\]

Here disease transitions are separated in time from demmographic
transitions \citep{Klepac2010}. In California, \emph{P. Ramorum}
reproduces and spreads during the winter rainy season while most tree
growth occurs in the spring and fall.

Where $\boldsymbol{A}$ is the matrix of demographic rates specifying
transitions between classes and mortality. $\boldsymbol{A}$ is a
block-diagonal matrix of size transition matrices, with no transitions
between species classes.

\begin{itemize}
\itemsep1pt\parskip0pt\parsep0pt
\item
  Build series of nested models of varying complexity that incorporate

  \begin{itemize}
  \itemsep1pt\parskip0pt\parsep0pt
  \item
    Species differences in demographic and epidemiological parameters
  \item
    Size structure in demographic and epidemiological parameters
  \item
    Spatial structure
  \end{itemize}
\item
  How does inclusion of each component of structure modify

  \begin{itemize}
  \itemsep1pt\parskip0pt\parsep0pt
  \item
    $R_0$ (global and local)
  \item
    Epidemic size
  \item
    Probability of outbreak
  \item
    Time to extinction of tanoak and \emph{P. ramorum}.
  \end{itemize}
\item
  Compare models to determine best predictor of disease outbreak
\item
  Determine optimal combination of host- (harvesting) and
  pathogen-centric (spraying) controls over management periods, given
  constraints, to minimize probability of disease outbreak
\end{itemize}

\subsection{Nested Models}

\begin{itemize}
\itemsep1pt\parskip0pt\parsep0pt
\item
  Fit all possible combinations of models:

  \begin{itemize}
  \itemsep1pt\parskip0pt\parsep0pt
  \item
    Including or excluding size structure of each species (by fixing
    parameters across classes)
  \item
    Including or excluding density dependence on reproduction, growth,
    and mortality
  \item
    Including or excluding disease effects on reproduction, growth, and
    mortality
  \item
    Normal, fat-tailed, or uniform (no spatial structure) dispersal
  \end{itemize}
\end{itemize}

Table 1: Model Structures

\begin{itemize}
\itemsep1pt\parskip0pt\parsep0pt
\item
  Mean field
\item
  Mean field with species classification
\item
  Mean field with size classification
\item
  Mean field with species and size classification
\item
  Spatially explicit
\item
  Spaitally explicit with species classification
\item
  SE with size class
\item
  SP with size class and species
\end{itemize}

Use \citet{Klepac2010} structural framework, extended to include
multiple species \emph{and} life stages.

\begin{itemize}
\itemsep1pt\parskip0pt\parsep0pt
\item
  Multinomial draws for transition stochasticity
\item
  Poission distributions for births
\item
  Environmental stochasticity - what does this most effect?

  \begin{itemize}
  \itemsep1pt\parskip0pt\parsep0pt
  \item
    Spore distribution/creation
  \end{itemize}
\end{itemize}

\subsection{Contact Structures}

Contact rates in SOD are not driven by differential mixing of groups,
but of 3D structure and differing physiological response of the disease.
Inter-species studies have mostly focused on (symmetrical) contact
structures where within-group contact rates exceed intra-group contact
rates, or assymetric contact rates simulating rare jumps between
species.

\begin{itemize}
\itemsep1pt\parskip0pt\parsep0pt
\item
  Observational stochasticity

  \begin{itemize}
  \itemsep1pt\parskip0pt\parsep0pt
  \item
    Binomial observation probability
  \item
    Ask Richard - what do we think false positive and negative rates are
    like?
  \end{itemize}
\end{itemize}

\subsection{Model Fitting and Selection}

In order to determine which model to use for planning and management, I
must select from the model structures above by determining which best
represents observations from time series data.

\subsection{Data}

Data to fit the models comes from a collaboration with the Rizzo lab's
diease monitoring plot network. The data consist of observations tree
size, condition, and disease status from the years 2002-2007. Trees were
observed in 14 sites along the California coast from X (36.16°N) to Y
(38.35°N), each of which contains X-X 500 m\textsuperscript{2},
seperated by minimium distances of Z. All plots are in forests dominated
by redwood, tanoak, and Bay Laurel, with negligible other SOD host
species.

Size, health, and disease status of all trees were measured in 2002 and
2007, as well as for a random sample of 5 trees of each species in the
years 2003-2006.

Each plot within a site will represent one subpopulation within the
metapopulation lattice, with the matrix of metapopulations around
counting as unobserved.

\subsection{Model fitting}

The processes generating the data above may be considered \textbf{hidden
Markov process} \citep{Gimenez2012}, that is, processes where the state
of the system is dependent on its previous state, but where states are
partially or imperfectly observed. Each model in Table 1 represents one
such process.

In order estimate the hidden Markov process, I require a method to
calculate.

\textbf{Particle filtering} determines the likelihood of a hidden Markov
process by taking advantage of the fact that, at each time step, the
likelihood of a system state is dependent on the previous state, and the
observation

p(X\_t) = p(X\_t \textbar{} X\_\{t-1\}, Y\_t)

The likelihood of the model, given the all of the data, may then be
expressed as

\[ \] \citep{Arulampalam2002}

Since $p(X_t \vert  X_{t-1})$ is unknown, It's value is determined at
each time step by simulation.

PFMCMC requires two components to the model. A \textbf{process model}
which simulates the states $(X_t)$ of the model, and an
\textbf{observation model} of the observations at each time point, given
underlying states $(Y_t \vert  X_t)$. Unlike the process model, the
observation model must have a tractable likelihood.

For my process model, I will assume that tree presence, size, and
location alive/dead status are measured exactly, but that disease status
is observed imperfectly via the conditional probabilites
\[ \begin{aligned}
 p_OI &= p(\text{Observing that tree is diseased}\vert \text{tree is diseased})
 p_OS &= p(\text{Observing that tree is healty}\vert \text{tree is healthy})
\end{aligned} \]

In addition, I assume that the smallest size class is unobserved in
models with size classes.

With an estimate of likelihood available, one needs an approach to
determine the maximum-likelihood estimate of the model parameters
$(\theta)$. Several methods are available. \textbf{Iterated filtering}
(IF) estimates $\theta$ by replacing constant parameter values with a
random walk $\theta_t$ of decreasing variance over time, and determining
the ``states'' of this walk as the variance approaches zero. IF is
computationally efficient but requires long time series.

Alternatively, \textbf{Particle Filter Markov Chain Monte Carlo}
(PFMCMC) uses a Markov Chain sampler (e.g., Metropolis-Hastings),
calculating the likelihood at each iteration using a particle filter.
While computationally expensive, it does not require long time series
and may be used with multiple, independent time series. PFMCMC also
integrates particle filtering into a Bayesian context, allowing the use
of informative priors.

\subsection{Model selection}

Model selection has multiple purposes. First, to determine which model
best is most like the true purpose, and to estimate the valdiity of the
models for the purposes of prediction. DIC has several advantages.
First, it incoporates any random variables. Secondly, it estimates the
loss of parameters from Bayesian inference. As several components of the
model have useul priors estimated form either data sets, this is useful
in this case.

PFMCMC will provide maximum-likelihood parameter estimates and
likelihoods for each of the model structures.

In order to compare them, I will use the \textbf{deviance information
criterion} \citep[DIC]{Spiegelhalter2002}, which is defined as

\[\text{DIC} = D(\bar\theta) + 2p_D\]

where $p_D$ is a metric of model complexity (effective number of
parameters) and $D$ is the deviance (-2 $\times$ log likelihood). $p_D$
defined as

\[p_D = \overline{D(\theta)} - D(\bar{\theta})\]

$p_D$ may be redfined as $\frac{p_D}{n}$ to correct for the amount of
data.

DIC provides a measure of fit and complexity. It has two primary
advantages over traditional measures of fit such as BIC and AIC. First,
since $p_D$ is based on the likelihood distribution, it incorporates
information from the prior. Effectively, it reduces the effective number
of parameters when they are restricted due to the prior. Secondly, the
same property allows the effective number of parameters from random
variables to be included.

Use of DIC assumes that there is a ``best'' or ``real'' model within the
comparison. While no models are ``true'' \citep{Box1976}, this excercise
is selecting a ``best'' or ``most useful'' model for scenario
exploration, rather than predictive model-averaging.

\begin{itemize}
\itemsep1pt\parskip0pt\parsep0pt
\item
  Compare models via AICc. Eliminate models with AICc weights
  \textless{} 4x the lowest AICc weight (following \citet{Maunder2011})
\end{itemize}

Informative priors for disease epidemiological parameters may be
determined from previous studies of SOD {[}@REFS{]}

\subsection{Optimization}

Using the model developed above I will derive optimal control strategies
for forest management under the threat of disease outbreak. The basis of
this work will be the \citet{Hartman1976} economic modeling framework,
in which forest owners are motivated by dual goals of continuous
ecosystem service amenities and profit from timber harvest. Under this
framework, the optimal control will minimize the costs associated with
quarantine and equipment cleaning, while maximizing both the standing
biomass of valued species and their timber value. Both time discounting
and risk-aversion will be included in the maximization scheme.

\citet{Reed1984} described a forestry model where age-dependent risk of
fire modifies the optimal harvest rotation of forest. Fire risk (a
nonhomogenous Poisson process) changes through time.

A stochastic implementation of the most parsimonious model will generate
a risk of outbreak as a function of forest composition and time. (See
Fig 1)

An outline of

\begin{itemize}
\item
  The best-fit model provides us with estimates of the probability over
  disease outbreaks in a forest stand over a management period. This
  probability is contingent on both forest composition and spatial
  structure, and the burden of spores from other areas, and changes over
  time as forest dynamics change the host population.

  By assuming that disease at below-outbreak levels has a negligible
  effect forest growth, this model can be simplified into a model of
  disease-free forest dynamics, and an evolving probability of outbreak
  that changes as a function of forest composition.

  By considering probability of disease outbreak a nonhomogenous poisson
  process, we may use the \citet{Reed1984} framework and extensions to
  calculate optimal control techniques.

  This becomes n optimal control problem in which $c_1$ is the annual
  cost of reducing the probability of disease arrival $\Lambda_t$, and
  $c_2$ is the cost of silvicultural treatment in any year of cutting.

  The goal will be to solve the dynamic optimization.

  Probability of outbreak will be reduced with lower populations of
  trees.

  Contraints:

  Budget (both annual and total scenarios) --\textgreater{} Minimize
  outbreak probability Maximum probability of outbreak --\textgreater{}
  Minimize cost
\item
  Both composition and the arrival of disease can be modified by
  controls. Select cutting may modify the composition and managment
  actions such as the cleaning of logging equipment can reduce the
  arrival of spores. Pesticides can also reduce the probability of
  spores successfully infecting healthy trees. All of these management
  actions (including logging) have costs, and are limited by budget
  constraints of management agencies and/or landowners
\item
  The solution to this optimal control problem will be an optimal
  \emph{treatment rotation} - a schedule of harvest and spraying actions
  that minimize risk over the treatment period.
\item
  An alternative useful formulation of the problem is to determine the
  minimum-cost method given constraints of desired tree population and
  \emph{maximum acceptable probability of outbreak}. An advantage of
  this approach is to provide managers with an method of implicit
  valuation of the tanoak population by illustrating the trade-off
  between costs and acceptable tree populations and risks.
\item
  Examine under a variety of regulatory constraints: limiting herbicide
  use for suppressing sprounts and others, harvest regulations in
  riparian areas, endangered species limitation on time/space of
  cutting.
\item
  An extension of the analysis above will include the parameter
  uncertainty of the best-fit model, and an optimization under the
  scenario that parameter uncertainties decrease with time
\end{itemize}

\begin{center}\rule{3in}{0.4pt}\end{center}

\section{Optimization}

The best-fit model that captures essential aspects of the disease
dynamics will be used to develop an optimal control solution to
minimization of disease risk over time.

In this, modeled after the XXXX area outbreak, goals for SOD forest
management can be approximated as (a) maintaining a minimum density of
at-risk species (i.e., tanoak), and (b) minimizing the probability of
outbreak.

Tools for SOD control can be categorized into two types. Silvicultural
(``host-centric'') treatments change forest composition.
``Pathogen-centric'' treatments, such as quarantine, equipment cleaning
and spraying, reduce the probability of spore arrival.

Cutting events are discrete, periodic treatments that change forest
composition. A cut can reduce the density of any class and species to
either the levels associated with minimum probability of outbreak, or to
minimum socially acceptable densities. Tanoak and bay laurel trees have
little timber value (@REF), so each cut has a one-time cost, $c_1$. For
the purpose of this model $c_1$ is independent of the extent of density
reduction.

$c_2$ represents the combined annual costs of measures to reduce the
external force of infection $\lambda_{\text{ex}}$.

This model could be optimized by defining a function for the value of
ecosystem services flowing annually from the forest annually as a
function of the forest composition $f(N_t)$ {[}@Hartmann1976{]}.
However, due to the difficulty of defining and valuing these services,
it is more feasible to define constraints that represent the range of
socially acceptable outcomes.

The constraint that makes most sense is a \emph{minimum acceeptable
density of the species of interest} $(N_{\min})$. This value could be
further specified to include, for instance, a minimum density of trees
of a minimum size. Since the system is stochastic, this constraint
expressed by a \emph{level of acceptable risk}, the probability that the
species density will fall beneath this value
$(p_{\text{acc}} > p(N_t < N_{\min})$.

The dynamic optimization problem is to minimize the cost of treatment
while maintaining the outbreak probability below $p_{\text{acc}}$:

\[\min \sum_1^T c_1 + c_2(\lambda_{ex}) \text{ s.t. } p(N_t < N_{\min}) < p_{text{acc}}\]

An alternative formulation, also useful to managers, is to minimize the
probability of outbreak given budget constraints. Budget constraints
could be forumulated as total budgets for the entire mangement period,
or annual limits on spending.

\[\min sum_1^T p_{\text{outbreak} \text{ s.t. } \sum_0^T c_1 + c_2(\lambda_{ex}) \leq B_T\]

or

\[\min sum_1^T p_{\text{outbreak} \text{ s.t. } c_1 + c_2(\lambda_{ex}) \leq B_{\text{ann}} \forall T\]
\renewcommand\refname{References}
\renewcommand\bibname{References}

%\begin{fullwidth}
\nobibliography{/Users/noamross/smallbib}
%\end{fullwidth}

\end{document}
